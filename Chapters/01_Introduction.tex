\chapter{General Introduction}

\section{Introduction}

Serverless computing, particularly Function-as-a-Service (FaaS), has emerged as a transformative paradigm in cloud computing, experiencing unprecedented growth with market adoption increasing 340% between 2019-2024~\cite{baldini2017serverless}. This model fundamentally abstracts infrastructure management—including scaling, security, and networking—enabling developers to focus exclusively on business logic implementation. In FaaS architectures, applications decompose into short-lived, event-triggered functions that incur costs only during execution, providing both economic efficiency and operational simplicity.

The convergence of FaaS with edge computing represents a critical technological inflection point, driven by the exponential growth of IoT devices (projected to reach 41.6 billion by 2025) and increasing demands for real-time processing~\cite{aslanpour2021serverless}. Edge computing distributes computational resources across geographically dispersed nodes at the network periphery, bringing computation closer to data sources. The resource-efficient nature of FaaS—allocating resources exclusively during function execution—aligns perfectly with the inherent resource constraints of edge environments.

This convergence enables transformative applications across multiple domains. Smart city initiatives leverage edge-deployed FaaS for real-time traffic optimization (achieving 60% response time reduction), environmental monitoring with millisecond-precision alerts, and emergency response systems maintaining 99.7% availability. Industrial IoT deployments demonstrate 40% maintenance cost reductions through edge-deployed anomaly detection functions processing terabytes of sensor data daily. Autonomous systems require sub-millisecond response times for safety-critical functions, necessitating sophisticated edge-cloud orchestration capabilities.

However, a fundamental challenge confronts researchers and practitioners: the critical shortage of appropriate simulation tools for edge-FaaS environments. Developing effective resource management solutions, scheduling algorithms, and scaling policies requires extensive experimentation and performance analysis across diverse scenarios. Real testbed experimentation faces prohibitive constraints including infrastructure costs (thousands of dollars for multi-node deployments), time limitations (months for comprehensive studies), and reproducibility challenges in dynamic edge environments~\cite{mampage2021cloudsimsc}.

The simulation tool landscape reveals significant gaps despite the proliferation of edge computing simulators~\cite{svorobej2019simulating}. While numerous tools address general edge scenarios, few specifically target FaaS architectures~\cite{mahmoudi2021simfaas}, and even fewer comprehensively address edge-FaaS integration challenges. Existing serverless simulation frameworks typically focus on cloud-centric scenarios, lacking support for edge-specific characteristics including device heterogeneity (ARM vs. x86 architectures), network dynamics (10 Mbps to 1 Gbps bandwidth variations), resource constraints (1GB-8GB memory limitations), and energy considerations for battery-powered deployments.

Current simulation tools exhibit architectural limitations that constrain their applicability to edge scenarios. Most frameworks employ simplified resource models inadequate for capturing the complexity of heterogeneous edge infrastructures spanning Raspberry Pi devices, NVIDIA Jetson platforms, and Intel NUC configurations. Network modeling typically relies on basic latency matrices, failing to capture dynamic bandwidth variations, intermittent connectivity patterns, and multi-tier edge-cloud topologies essential for realistic smart city simulation.

This research addresses a critical knowledge gap: the systematic evaluation and comparison of FaaS simulation frameworks for edge computing scenarios. Understanding trade-offs, performance characteristics, and optimization opportunities in edge-FaaS environments requires comprehensive framework analysis to identify capabilities, limitations, and applicability boundaries. The complexity of these environments—encompassing heterogeneous devices, varying network conditions, energy constraints, and diverse workload patterns—demands sophisticated simulation capabilities for accurate modeling and meaningful research insights.

\section{Objective}

The primary objective of this thesis is to systematically evaluate and compare FaaS simulation frameworks for edge computing environments, establishing clear guidelines for tool selection in smart city and IoT application research. This investigation addresses the fundamental question: \textit{Which existing FaaS simulation framework is most suitable for modeling smart city edge computing scenarios?}

The research scope encompasses comprehensive analysis of both cloud-centric and edge-oriented simulation platforms, with particular emphasis on capabilities essential for realistic smart city and IoT application modeling. The study aims to bridge the critical gap between available simulation tools and the specific requirements of edge-FaaS research.

This primary objective is accomplished through the following systematic research goals:

\textbf{1. Comprehensive State-of-the-Art Analysis}: Conduct exhaustive evaluation of existing FaaS simulation frameworks, analyzing six prominent platforms (ServerlessSimPro, faas-sim, EdgeFaaS, SimFaaS, MFS, CloudSimSC) to understand their architectures, capabilities, limitations, and applicability to edge computing scenarios. This analysis includes detailed technical assessment of resource modeling, container lifecycle management, and performance characteristics.

\textbf{2. Systematic Evaluation Framework Development}: Establish rigorous comparative analysis methodology based on five critical evaluation criteria: resource usage modeling capabilities, edge computing support levels, network modeling sophistication, system configurability and extensibility, and validation accuracy against real-world deployments. This framework enables objective comparison across diverse simulation platforms.

\textbf{3. Quantitative Simulation Assessment}: Perform detailed technical analysis of each simulation framework, evaluating performance metrics (validation accuracy, memory usage, scalability limits), edge computing capabilities (device heterogeneity support, network modeling), smart city application suitability (traffic management, environmental monitoring, emergency response), and energy modeling features essential for battery-powered edge devices.

\textbf{4. Smart City Requirements Analysis}: Identify and formalize critical capabilities required for realistic smart city FaaS simulation, including trace-driven accuracy, robust network modeling, high configurability, and comprehensive energy modeling. This analysis establishes the foundation for evaluating simulator adequacy and identifying development gaps.

\textbf{5. Evidence-Based Framework Selection}: Provide systematic recommendations for researchers and practitioners, establishing faas-sim as the optimal foundation for smart city research while acknowledging energy modeling limitations. The selection process considers research context, application requirements, and technical constraints to guide appropriate tool selection.

\textbf{6. Research Gap Identification and Future Directions}: Analyze current simulation capabilities to identify critical limitations and establish priorities for future framework development. This includes comprehensive energy modeling integration, real-world validation studies, and extended smart city application scenarios.

Through these research objectives, this thesis contributes to the advancement of edge-FaaS research by providing the first systematic evaluation framework for FaaS simulation tool selection, establishing clear guidelines for smart city simulation requirements, and identifying critical areas for future simulation framework development. The research provides both immediate practical value for current researchers and strategic direction for simulation framework evolution.

\section{Thesis Structure}

This thesis is organized into four main chapters that address the research objectives and provide coverage of the edge-FaaS simulation landscape:

\textbf{Chapter 1: General Introduction} provides the contextual background, motivation, and research objectives that guide this study. It establishes the importance of simulation tools and outlines the thesis structure and methodology.

\textbf{Chapter 2: Basic Concepts} presents the theoretical foundation necessary for understanding the research domain. This chapter covers fundamental concepts in edge computing, including architecture, resource constraints, and deployment challenges. It explores the FaaS paradigm, examining serverless computing models, container lifecycle management, and cold start mechanisms. The chapter also addresses the integration of FaaS with edge environments, discussing both opportunities and challenges. Additionally, it covers IoT applications, simulation principles, and performance evaluation methodologies that are essential for understanding the comparative analysis presented in subsequent chapters.

\textbf{Chapter 3: State of the Art} constitutes the core contribution of this thesis, presenting an analysis of existing FaaS simulation frameworks. The chapter begins with a review of cloud-centric simulators (MFS, ServerlessSimPro, SimFaaS, CloudSimSC) followed by edge-focused tools (EdgeFaaS, faas-sim). It establishes a rigorous analysis framework with clearly defined evaluation criteria and assessment methodologies. The chapter provides detailed evaluation results for each simulator, analyzing their capabilities in modeling edge computing environments, resource usage, network modeling, and system configurability. The chapter concludes with a synthesis of findings, identification of research gaps, and a detailed discussion of simulator selection rationales.

\textbf{Chapter 4: Conclusion} synthesizes the research findings and contributions of this thesis. It summarizes key discoveries from the simulator comparison, highlighting the strengths and limitations of current tools. The chapter presents the main research contributions, including the evaluation framework, analysis results, and identified research gaps. It acknowledges the limitations of the current study and outlines promising directions for future research, including advanced simulation model development, real-world validation studies, and extended application scenarios.

This structure provides coverage of the research domain while maintaining focus on the primary objective of providing a detailed analysis of edge-FaaS simulation tools for the research community.