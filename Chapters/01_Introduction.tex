\chapter{General Introduction}

\section{Introduction}

The combination of serverless computing and edge computing represents a major shift in modern distributed systems. Serverless computing, especially through the Function-as-a-Service (FaaS) model, has changed cloud service delivery by removing infrastructure management complexity~\cite{baldini2017serverless}. This allows developers to focus only on business logic implementation. In the FaaS model, applications are broken down into short-lived, event-triggered functions that use resources only during execution, providing excellent cost efficiency and scalability benefits.

The rise of artificial intelligence (AI) and machine learning applications has further accelerated FaaS adoption. AI functions, such as image recognition, natural language processing, and predictive analytics, are particularly well-suited to the FaaS model due to their event-driven nature and variable computational demands. These AI workloads benefit tremendously from the automatic scaling and pay-per-use pricing that FaaS provides, making advanced AI capabilities accessible to a broader range of applications and developers.

Edge computing addresses the limitations of centralized cloud computing by bringing computational resources closer to data sources and end users. This approach reduces latency, improves privacy and security, and enables real-time processing capabilities needed for modern Internet of Things (IoT) applications. The integration of FaaS with edge computing environments creates exciting opportunities for deploying AI-powered functions directly at the network edge, enabling intelligent real-time decision making where data is generated.

The application of FaaS principles to edge computing has gained significant interest from both researchers and industry practitioners~\cite{aslanpour2021serverless}. Edge nodes have limited resources, which makes them benefit greatly from the FaaS model's efficient resource allocation strategy. Resources are provided dynamically based on actual demand. This alignment is especially valuable for AI-powered IoT applications and smart city scenarios, where intelligent functions such as computer vision for traffic monitoring, speech recognition for emergency response, or predictive maintenance algorithms need to process data in real-time with minimal latency.

The potential for deploying AI functions at the edge opens up revolutionary possibilities: autonomous vehicles making split-second decisions, smart cameras detecting anomalies instantly, or IoT sensors predicting equipment failures before they occur.

However, a critical challenge facing researchers and practitioners seeking to explore edge-FaaS deployments is the lack of appropriate simulation tools. While many simulation frameworks exist for traditional edge computing scenarios~\cite{svorobej2019simulating} and cloud-based FaaS systems~\cite{mahmoudi2021simfaas}, very few tools specifically address the unique requirements and constraints of FaaS applications operating in edge environments~\cite{jeon2019cloudsim}. This gap significantly limits research progress and prevents the development of effective edge-FaaS solutions.

The complexity of edge-FaaS environments includes heterogeneous devices, varying network conditions, resource constraints, and diverse workload patterns. This requires sophisticated simulation capabilities that can accurately model these complex interactions. Understanding the trade-offs, performance characteristics, and optimization opportunities in such environments requires comprehensive evaluation frameworks that existing tools do not provide adequately.

\section{Objective}

The primary objective of this thesis is to conduct a comprehensive analysis and comparison of existing simulation frameworks for Function-as-a-Service applications deployed in edge computing environments, with particular emphasis on their suitability for smart city IoT scenarios.

This main goal is achieved through the following specific research objectives:

\textbf{1. Comprehensive State-of-the-Art Analysis}: Conduct a systematic survey of existing FaaS simulation tools, analyzing both cloud-centric and edge-oriented frameworks to understand their capabilities, limitations, and applicability to edge-FaaS scenarios.

\textbf{2. Evaluation Framework Development}: Establish a comprehensive evaluation methodology based on critical criteria including edge computing support, IoT workload modeling capabilities, resource usage tracking, network modeling accuracy, and system configurability.

\textbf{3. Comparative Simulation Assessment}: Perform detailed comparative analysis of identified simulation frameworks, evaluating their strengths and weaknesses across the established criteria to determine their suitability for different edge-FaaS research scenarios.

\textbf{4. Smart City Use Case Analysis}: Investigate the specific requirements and characteristics of smart city IoT applications in the context of edge-FaaS deployments, identifying the simulation capabilities necessary to model such scenarios effectively.

\textbf{5. Research Gap Identification}: Identify limitations in current simulation tools and highlight areas where further development is needed to advance edge-FaaS research and development.

\textbf{6. Framework Selection }: Provide evidence-based recommendations for researchers and practitioners seeking appropriate simulation tools for edge-FaaS studies, considering different research objectives and application domains.

Through these objectives, this thesis aims to contribute to the advancement of edge-FaaS research by providing a systematic foundation for tool selection and highlighting critical areas for future simulation framework development.

\section{Thesis Structure}

This thesis is organized into four main chapters that systematically address the research objectives and provide comprehensive coverage of the edge-FaaS simulation landscape:

\textbf{Chapter 1: General Introduction} provides the contextual background, motivation, and research objectives that guide this study. It establishes the importance of edge-FaaS simulation tools and outlines the thesis structure and methodology.

\textbf{Chapter 2: Basic Concepts} presents the theoretical foundation necessary for understanding the research domain. This chapter covers fundamental concepts in edge computing, including architecture, resource constraints, and deployment challenges. It explores the FaaS paradigm, examining serverless computing models, container lifecycle management, and cold start mechanisms. The chapter also addresses the integration of FaaS with edge environments, discussing both opportunities and challenges. Additionally, it covers IoT and smart city applications, simulation principles, and performance evaluation methodologies that are essential for understanding the comparative analysis presented in subsequent chapters.

\textbf{Chapter 3: State of the Art} constitutes the core contribution of this thesis, presenting a comprehensive analysis of existing FaaS simulation frameworks. The chapter begins with a systematic review of both cloud-centric simulators (MFS, ServerlessSimPro, SimFaaS, CloudSimSC) and edge-oriented tools (EdgeFaaS, faas-sim). It establishes a rigorous comparative analysis framework with clearly defined evaluation criteria and assessment methodologies. The chapter provides detailed evaluation results for each simulator, analyzing their capabilities in modeling edge computing environments, IoT workloads, resource usage, network dynamics, and system configurability. It examines smart city IoT applications in the context of edge-FaaS deployments and analyzes the current research landscape. The chapter concludes with a synthesis of findings, identification of research gaps, and a detailed discussion of simulator selection rationales.

\textbf{Chapter 4: Conclusion} synthesizes the research findings and contributions of this thesis. It summarizes key discoveries from the comprehensive simulator comparison, highlighting the strengths and limitations of current tools. The chapter presents the main research contributions, including the comprehensive evaluation framework, comparative analysis results, and identified research gaps. It acknowledges the limitations of the current study and outlines promising directions for future research, including advanced simulation model development, real-world validation studies, and extended application scenarios.

This structure ensures systematic coverage of the research domain while maintaining focus on the primary objective of providing a comprehensive comparative analysis of edge-FaaS simulation tools for the research community.