\part{General Introduction}
\chapter{General Introduction}

\section{Introduction}

Serverless computing, particularly Function-as-a-Service (FaaS), has emerged as a transformative paradigm in cloud computing, experiencing unprecedented growth with the market projected to expand 340\% between 2023-2031~\cite{datam2024serverless}. This model fundamentally removes the burden of infrastructure management: scaling, security, and networking, allowing developers to focus exclusively on business logic implementation~\cite{baldini2017serverless}. In FaaS architectures, applications decompose into short-lived, event-triggered functions that incur costs only during execution, with enterprise adoption reaching remarkable levels where over 70\% of AWS customers now utilize serverless solutions~\cite{datadog2023serverless}.

Applying the FaaS model to edge computing has recently received significant interest in research as well as industry~\cite{aslanpour2021serverless}. Edge computing refers to taking advantage of computational nodes located at the edge of network. By allocating resources only during function execution, FaaS assures reduced resource consumption, which is essential for the resource-constrained nature of edge nodes. The potential for deploying AI functions at the edge opens up exciting use cases such as cameras detecting anomalies, IoT sensors predicting equipment failures before they occur, and agriculture monitoring systems analyzing soil conditions, all with minimal latency and bandwidth usage.

A main challenge for researchers and practitioners seeking to deploy the FaaS model to edge scenarios is the shortage of simulation tools. Designing and building effective resource management solutions for serverless environments requires extensive long-term testing, experimentation, and analysis of performance metrics. However, utilizing real test beds and serverless platforms for such experimentation is often not feasible due to resource, time, and cost constraints~\cite{mampage2021cloudsimsc}. This challenge is particularly acute when evaluating novel ideas for resource management, function scheduling, and scaling policies in edge-FaaS environments.

Specifically, although there is a wide range of simulation tools targeting edge and fog scenarios~\cite{svorobej2019simulating}, there are only a few tools focusing on the FaaS model~\cite{mahmoudi2021simfaas} and, in particular, on executing FaaS applications in edge environments. Existing simulation software developed for serverless environments often lack generalizability in their architecture and fail to comprehensively address various aspects of resource management, with most being purely focused on modeling function performance under specific platform architectures~\cite{mampage2021cloudsimsc}. Moreover, these tools typically do not provide adequate support for the unique characteristics of edge computing, such as network latency variations, resource heterogeneity, and intermittent connectivity patterns.

The complexity of this environment includes heterogeneous devices, varying network conditions, resource constraints, and diverse workload patterns. This requires sophisticated simulation capabilities that can accurately model these interactions. Understanding the trade-offs, performance characteristics, and optimization opportunities in such environments requires comprehensive evaluation of the frameworks to identify what existing tools miss and what they do not provide adequately.

\section{Objective}

The aim of this thesis is to explore simulators for FaaS deployed at the edge, Analyze them  with emphasis on factors that make them most suitable for edge scenarios.

This main goal is achieved through the following research objectives:

\textbf{1. State-of-the-Art Analysis}: Conduct a survey of existing FaaS simulation tools, analyzing both cloud-centric and edge-oriented frameworks to understand their features, architectures, limitations, and applicability to edge scenarios.

\textbf{2. Comparative Study Framework}: Establish a comparative study based on critical factors including edge computing support, IoT workload modeling capabilities, resource usage tracking, network modeling accuracy, and system configurability.

\textbf{3. Simulation Assessment}: Perform detailed analysis of identified simulation frameworks, evaluating their strengths and weaknesses across the established criteria to determine their suitability for different edge-FaaS research scenarios.

\textbf{4. Framework Selection}: Provide evidence-based recommendations for researchers and practitioners seeking appropriate simulation tools for edge-FaaS studies, considering different research objectives and application domains.

Through these objectives, this thesis aims to contribute to the advancement of edge-FaaS research by providing a foundation for tool selection and highlighting critical areas for future simulation framework development.

\section{Thesis Structure}

This thesis is organized into four main chapters that address the research objectives and provide coverage of the edge-FaaS simulation landscape:

\textbf{Chapter 1: General Introduction} provides the contextual background, motivation, and research objectives that guide this study. It establishes the importance of simulation tools and outlines the thesis structure and methodology.

\textbf{Chapter 2: Basic Concepts} presents the theoretical foundation necessary for understanding the research domain. This chapter covers fundamental concepts in edge computing, including architecture, resource constraints, and deployment challenges. It explores the FaaS paradigm, examining serverless computing models, container lifecycle management, and cold start mechanisms. The chapter also addresses the integration of FaaS with edge environments, discussing both opportunities and challenges. Additionally, it covers IoT applications, simulation principles, and performance evaluation methodologies that are essential for understanding the comparative analysis presented in subsequent chapters.

\textbf{Chapter 3: State of the Art} constitutes the core contribution of this thesis, presenting an analysis of existing FaaS simulation frameworks. The chapter begins with a review of cloud-centric simulators (MFS, ServerlessSimPro, SimFaaS, CloudSimSC) followed by edge-focused tools (EdgeFaaS, faas-sim). It establishes a rigorous analysis framework with clearly defined evaluation criteria and assessment methodologies. The chapter provides detailed evaluation results for each simulator, analyzing their capabilities in modeling edge computing environments, resource usage, network modeling, and system configurability. The chapter concludes with a synthesis of findings, identification of research gaps, and a detailed discussion of simulator selection rationales.

\textbf{Chapter 4: Conclusion} synthesizes the research findings and contributions of this thesis. It summarizes key discoveries from the simulator comparison, highlighting the strengths and limitations of current tools. The chapter presents the main research contributions, including the evaluation framework, analysis results, and identified research gaps. It acknowledges the limitations of the current study and outlines promising directions for future research, including advanced simulation model development, real-world validation studies, and extended application scenarios.

This structure provides coverage of the research domain while maintaining focus on the primary objective of providing a detailed analysis of edge-FaaS simulation tools for the research community.