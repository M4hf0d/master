% Abstract
\vspace*{4cm}
\begin{center}
    {\Large\bf Abstract}
\end{center} \vskip 0.5cm

Serverless computing, particularly Function-as-a-Service (FaaS), has gained significant adoption in edge computing environments for IoT and smart city applications. However, the lack of appropriate simulation tools presents challenges for evaluating FaaS-edge deployments before real-world implementation.

This thesis investigates which existing FaaS simulation framework is most suitable for simulating FaaS at the edge, with specific focus on smart city scenarios. We evaluated six prominent simulation frameworks: ServerlessSimPro, faas-sim, EdgeFaaS, SimFaaS, MFS, and CloudSimSC using five evaluation criteria including edge computing support, network modeling, configurability, and validation accuracy.

The evaluation reveals that while no existing simulator comprehensively addresses all edge FaaS requirements within a single platform, however we found one that provides the most promising foundation for future development. Cloud-centric simulators lack edge-specific capabilities, while edge-oriented simulators have limitations in energy modeling and comprehensive metrics. The study identifies four essential requirements for edge FaaS simulation and establishes an evaluation methodology for simulator selection.

The research contributes a systematic evaluation framework for FaaS simulators and provides recommendations for future development of edge FaaS simulation capabilities. This work establishes guidelines for advancing simulation tools in serverless edge computing research.

\providecommand{\keywords}[1] {
  \small
  \textbf{\textit{Keywords---}} #1
}
\keywords{Serverless Computing, Function-as-a-Service, Edge Computing, Smart Cities, Simulation Frameworks, Internet of Things, Network Modeling, Energy Modeling}

\newpage

% French Abstract
\vspace*{4cm}
\begin{center}
    {\Large\bf Résumé}
\end{center} \vskip 0.5cm

L'informatique sans serveur, en particulier la Function-as-a-Service (FaaS), a gagné une adoption significative dans les environnements de périphérie pour les applications IoT et villes intelligentes. Cependant, le manque d'outils de simulation appropriés présente des défis pour l'évaluation des déploiements FaaS-edge avant l'implémentation réelle.

Cette thèse examine quel framework de simulation FaaS existant est le plus adapté pour simuler FaaS à la périphérie, avec un focus spécifique sur les scénarios de villes intelligentes. Nous avons évalué systématiquement six frameworks de simulation proéminents : ServerlessSimPro, faas-sim, EdgeFaaS, SimFaaS, MFS, et CloudSimSC en utilisant cinq critères d'évaluation incluant le support de l'informatique de périphérie, la modélisation réseau, la configurabilité, et la précision de validation.

L'évaluation révèle que bien qu'aucun simulateur existant ne réponde de manière exhaustive à toutes les exigences FaaS de périphérie au sein d'une seule plateforme, nous avons trouvé celui qui fournit la base la plus prometteuse pour le développement futur. Les simulateurs centrés sur le cloud manquent de capacités spécifiques à la périphérie, tandis que les simulateurs orientés périphérie ont des limitations en modélisation énergétique et métriques complètes. L'étude identifie quatre exigences essentielles pour la simulation FaaS de périphérie et établit une méthodologie d'évaluation pour la sélection de simulateurs.

La recherche contribue un framework d'évaluation systématique pour les simulateurs FaaS et fournit des recommandations pour le développement futur des capacités de simulation FaaS de périphérie. Ce travail établit des directives pour faire progresser les outils de simulation dans la recherche en informatique sans serveur de périphérie.

Les directions de recherche future incluent l'implémentation de modélisation énergétique complète au sein de faas-sim, la conduite d'études de validation empiriques comparant les prédictions de simulation avec les déploiements réels de périphérie, et l'expansion des capacités de simulation à divers scénarios informatique de périphérie incluant les villes intelligentes, l'IoT industriel, et les systèmes autonomes. Alors que les technologies d'informatique de périphérie continuent de progresser, les frameworks de simulation évalués à travers cette recherche joueront des rôles cruciaux dans le développement d'applications sans serveur distribuées de nouvelle génération.

\providecommand{\keywordsfr}[1] {
  \small
  \textbf{\textit{Mots-clés---}} #1
}
\keywordsfr{Informatique sans Serveur, Function-as-a-Service, Informatique de Périphérie, Villes Intelligentes, Frameworks de Simulation, Internet des Objets, Modélisation Réseau, Modélisation Énergétique}

\newpage
\selectlanguage{arabic}
\vspace*{4cm}
\begin{center}
    {\Large\bf ملخص}
\end{center} \vskip 0.5cm

\selectlanguage{arabic}
لقد حققت الحوسبة اللاخادمية، وخاصة الوظائف كخدمة (FaaS)، انتشاراً واسعاً في بيئات الحوسبة الطرفية لتطبيقات إنترنت الأشياء والمدن الذكية. ومع ذلك، فإن نقص أدوات المحاكاة المناسبة يطرح تحديات لتقييم نشر FaaS-edge قبل التطبيق الفعلي.

تبحث هذه الأطروحة في أي إطار عمل محاكاة FaaS موجود هو الأنسب لمحاكاة FaaS في الحوسبة الطرفية، مع التركيز بشكل خاص على سيناريوهات المدن الذكية. قمنا بتقييم منهجي لستة أطر عمل محاكاة بارزة: ServerlessSimPro، faas-sim، EdgeFaaS، SimFaaS، MFS، وCloudSimSC باستخدام خمسة معايير تقييم تشمل دعم الحوسبة الطرفية، نمذجة الشبكة، القابلية للتكوين، ودقة التحقق.

يكشف التقييم أنه بينما لا يوجد محاكي حالي يلبي بشكل شامل جميع متطلبات FaaS الطرفية ضمن منصة واحدة، إلا أننا وجدنا الذي يوفر الأساس الأكثر واعدية للتطوير المستقبلي. تفتقر المحاكيات المركزة على السحابة إلى القدرات الخاصة بالحوسبة الطرفية، بينما تواجه المحاكيات الموجهة للحوسبة الطرفية قيوداً في النمذجة الطاقوية والمقاييس الشاملة. تحدد الدراسة أربعة متطلبات أساسية لمحاكاة FaaS الطرفية وتؤسس منهجية تقييم لاختيار المحاكيات.

يساهم البحث بإطار تقييم منهجي لمحاكيات FaaS ويقدم توصيات للتطوير المستقبلي لقدرات محاكاة FaaS الطرفية. يضع هذا العمل إرشادات لتطوير أدوات المحاكاة في بحوث الحوسبة اللاخادمية الطرفية.

\providecommand{\keywordsar}[1] {
  \small
  \textbf{\textit{الكلمات المفتاحية---}} #1
}
\keywordsar{الحوسبة اللاخادمية، الوظائف كخدمة، الحوسبة الطرفية، المدن الذكية، أطر عمل المحاكاة، إنترنت الأشياء، نمذجة الشبكة، النمذجة الطاقوية}
\selectlanguage{english}
