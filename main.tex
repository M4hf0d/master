% Master's Thesis: Simulating FaaS Applications at the Edge
% Following the 4-Chapter Template Structure
\documentclass[12pt,a4paper]{report}

% Package imports
\usepackage[utf8]{inputenc}
\usepackage[T1]{fontenc}
\usepackage{geometry}
\usepackage{fancyhdr}
\usepackage{enumitem}
\usepackage{titlesec}
\usepackage{booktabs}

% Page setup
\geometry{left=3cm,right=2cm,top=2.5cm,bottom=2.5cm}

\begin{document}

% ==================== THESIS STRUCTURE ====================

\title{Simulating FaaS Applications at the Edge for IoT Workloads: \\
Energy-Aware Scaling Strategies in Heterogeneous Computing Environments}
\author{Mahfoud Abd el Ali SAYAH}
\date{Academic Year 2024-2025}

\maketitle
\tableofcontents

% ==================== 4-CHAPTER STRUCTURE ====================

\chapter{General Introduction}
\textit{(8-10 pages)}

\section{Introduction}
\begin{itemize}[leftmargin=1cm]
    \item Context of serverless computing growth and FaaS popularity
    \item Edge computing paradigm and resource-constrained environments
    \item Convergence opportunity: FaaS at the edge for IoT applications
    \item Main challenge: shortage of simulation tools for edge FaaS deployments
\end{itemize}

\section{Objective}
\textbf{Primary Goal:} Develop energy-aware scaling framework for FaaS applications in edge computing environments through simulation

\textbf{Specific Objectives:}
\begin{enumerate}[leftmargin=1cm]
    \item Conduct state-of-the-art analysis on edge computing simulators and FaaS tools
    \item Identify suitable simulation framework and extend with energy modeling
    \item Implement energy-aware scaling strategies for heterogeneous edge devices
    \item Validate framework through IoT smart city use case and comparative evaluation
    \item Provide deployment recommendations for edge-native serverless computing
\end{enumerate}

\section{Dissertation Structure}
\begin{itemize}[leftmargin=1cm]
    \item Chapter 1: Research context, objectives, and methodology overview
    \item Chapter 2: Foundational concepts in edge computing, FaaS, and energy optimization
    \item Chapter 3: State-of-the-art analysis and simulation framework selection
    \item Chapter 4: Implementation, experiments, and results analysis
\end{itemize}

% ==================== FOUNDATIONAL CONCEPTS ====================

\chapter{Basic Concepts}
\textit{(18-22 pages)}

\section{Introduction}

\section{Edge Computing Fundamentals}
\begin{itemize}[leftmargin=1cm]
    \item Definition and key characteristics
    \item Edge vs. cloud computing paradigms
    \item Resource constraints and deployment challenges
    \item Heterogeneous device ecosystems
\end{itemize}

\section{Serverless Computing and FaaS}
\begin{itemize}[leftmargin=1cm]
    \item Serverless computing evolution
    \item Function-as-a-Service model characteristics
    \item Event-driven execution and automatic scaling
    \item Benefits and limitations for edge deployment
\end{itemize}

\section{FaaS at the Edge}
\begin{itemize}[leftmargin=1cm]
    \item Opportunities and challenges
    \item Resource allocation considerations
    \item Cold start and warm start implications
    \item Network latency and bandwidth constraints
\end{itemize}

\section{Heterogeneous Edge Devices}
\begin{itemize}[leftmargin=1cm]
    \item Device taxonomy (RPi, Coral TPU, Jetson devices, Intel NUC)
    \item Performance and energy characteristics
    \item Specialized accelerators and general-purpose compute
    \item Real-world deployment distributions
\end{itemize}

\section{Energy Models and Optimization}
\begin{itemize}[leftmargin=1cm]
    \item Power consumption models (idle vs. dynamic)
    \item Energy efficiency metrics
    \item Performance vs. energy trade-offs
    \item Battery-powered device considerations
\end{itemize}

\section{Autoscaling and Resource Management}
\begin{itemize}[leftmargin=1cm]
    \item Traditional cloud scaling approaches
    \item Edge-specific scaling challenges
    \item Load balancing in heterogeneous environments
    \item Quality of Service requirements
\end{itemize}

\section{IoT Applications and Smart Cities}
\begin{itemize}[leftmargin=1cm]
    \item IoT application characteristics
    \item Smart city infrastructure requirements
    \item Traffic monitoring use case
    \item Real-time processing constraints
\end{itemize}

\section{Conclusion}

% ==================== STATE OF THE ART ====================

\chapter{State of the Art}
\textit{(15-18 pages)}

\section{Introduction}

\section{Edge Computing Simulation Tools}
\begin{itemize}[leftmargin=1cm]
    \item CloudSim and EdgeCloudSim
    \item iFogSim and YAFS (Yet Another Fog Simulator)
    \item Limitations for FaaS workloads
    \item Gap analysis for serverless simulation
\end{itemize}

\section{FaaS Simulation Frameworks}
\begin{itemize}[leftmargin=1cm]
    \item SimFaaS: Performance simulator for serverless platforms
    \item CloudSim extensions for FaaS
    \item FAAS-Profiler and performance modeling
    \item Limitations for edge environments
\end{itemize}

\section{Edge-FaaS Simulation Tools}
\begin{itemize}[leftmargin=1cm]
    \item CloudSim extension for distributed functions
    \item HeroSim and existing solutions
    \item \textbf{faas-sim framework identification and selection}
    \item Requirements for energy modeling extension
\end{itemize}

\section{Energy-Aware Computing Approaches}
\begin{itemize}[leftmargin=1cm]
    \item Single-objective energy optimization
    \item Multi-objective optimization approaches
    \item Scaling strategies in literature
\end{itemize}

\section{Research Gap Analysis}
\begin{itemize}[leftmargin=1cm]
    \item Shortage of edge-FaaS simulation tools
    \item Lack of energy modeling in existing frameworks
    \item Missing scaling strategies for heterogeneous devices
    \item Limited IoT use case validation
\end{itemize}

\section{Framework Selection Justification}
\begin{itemize}[leftmargin=1cm]
    \item Multi-criteria evaluation of existing tools
    \item faas-sim selection rationale
    \item Extension requirements identification
    \item Energy model integration strategy
\end{itemize}

\section{Conclusion}

% ==================== CONTRIBUTION/IMPLEMENTATION ====================

\chapter{Our Contribution}
\textit{(25-30 pages)}

\section{Introduction}
Overview of research contributions and implementation approach

\section{Framework Architecture and Extensions}

\subsection{faas-sim Framework Analysis}
\begin{itemize}[leftmargin=1cm]
    \item Original framework capabilities and limitations
    \item Architecture analysis and extension points
    \item Energy modeling gap identification
\end{itemize}

\subsection{Energy Model Integration}
\begin{itemize}[leftmargin=1cm]
    \item Linear power consumption model implementation
    \item Device-specific energy profiles
    \item Dynamic vs. static power modeling
    \item Validation against manufacturer specifications
\end{itemize}

\section{Heterogeneous Edge Device Modeling}

\subsection{Device Distribution Strategy}

\begin{table}[h]
\centering
\caption{Implemented Edge Device Distribution}
\begin{tabular}{lll}
\toprule
\textbf{Device Type} & \textbf{Percentage} & \textbf{Primary Use Case} \\
\midrule
Raspberry Pi & 35\% & IoT sensors, basic monitoring \\
RockPi & 25\% & General compute nodes \\
Coral TPU & 15\% & AI inference acceleration \\
Jetson Nano & 12\% & GPU-based inference \\
Jetson NX & 8\% & High-performance AI \\
Intel NUC & 2.5\% & Edge coordinators \\
Jetson TX2 & 2.5\% & Specialized workloads \\
\bottomrule
\end{tabular}
\end{table}

\subsection{Power Consumption Implementation}
\begin{itemize}[leftmargin=1cm]
    \item Device-specific power profiles
    \item Idle and maximum power consumption values
    \item Utilization-based dynamic power calculation
    \item Energy efficiency metrics calculation
\end{itemize}

\section{Energy-Aware Scaling Strategies}

\subsection{Base Autoscaler Framework}
\begin{itemize}[leftmargin=1cm]
    \item Template method pattern implementation
    \item Common scaling decision logic
    \item Metrics collection system
    \item Strategy interface design
\end{itemize}

\subsection{High Performance Short Time (HPST)}
\begin{itemize}[leftmargin=1cm]
    \item Philosophy: "high power × short time"
    \item Device prioritization hierarchy
    \item Implementation details and algorithm
    \item Use case: performance-critical applications
\end{itemize}

\subsection{Low Power Long Time (LPLT)}
\begin{itemize}[leftmargin=1cm]
    \item Philosophy: "low power × long time"
    \item Energy-first device selection
    \item Battery-aware considerations
    \item Use case: energy-constrained deployments
\end{itemize}

\subsection{Kubernetes-Style First Fit}
\begin{itemize}[leftmargin=1cm]
    \item Weighted scoring system implementation
    \item Device class matching logic
    \item Resource balancing approach
    \item Production baseline compatibility
\end{itemize}

\subsection{Standard First Fit (Baseline)}
\begin{itemize}[leftmargin=1cm]
    \item Resource-only allocation
    \item Comparison baseline implementation
    \item Traditional container orchestration approach
\end{itemize}

\section{Smart City Use Case Implementation}

\subsection{IoT Traffic Monitoring Scenario}
\begin{itemize}[leftmargin=1cm]
    \item Function types: resnet50-inference, speech-inference, etc.
    \item Initial deployment strategy (light deployment)
    \item Geographic distribution modeling
    \item Realistic workload patterns
\end{itemize}

\subsection{Function Deployment Configuration}
\begin{itemize}[leftmargin=1cm]
    \item 16 initial function instances across 6 types
    \item Cold-start problem mitigation
    \item Performance requirements per function
    \item Resource allocation specifications
\end{itemize}

\section{Implementation and Development}

\subsection{Development Environment}
\begin{itemize}[leftmargin=1cm]
    \item Programming languages (Python, etc.)
    \item Framework modifications and extensions
    \item Code organization and architecture
\end{itemize}

\subsection{Simulation Configuration}
\begin{itemize}[leftmargin=1cm]
    \item Experimental parameters
    \item Device generation and deployment
    \item Workload simulation setup
    \item Metrics collection implementation
\end{itemize}

\section{Evaluation Methodology}

\subsection{Experimental Design}
\begin{itemize}[leftmargin=1cm]
    \item Comparative evaluation approach
    \item Multiple simulation runs
    \item Statistical analysis methods
    \item Performance vs. energy metrics
\end{itemize}

\subsection{Evaluation Metrics}
\begin{itemize}[leftmargin=1cm]
    \item Energy Efficiency Score: requests/joule
    \item Response time percentiles
    \item Resource utilization patterns
    \item Scaling decision frequency
\end{itemize}

\section{Results and Analysis}
\subsection{Energy Consumption Results}

\begin{table}[h]
\centering
\caption{Energy Consumption Comparison}
\begin{tabular}{llll}
\toprule
\textbf{Strategy} & \textbf{Total Energy (kJ)} & \textbf{Energy/Request (J)} & \textbf{Improvement} \\
\midrule
LPLT & 1,156 & 2.68 & 16.8\% \\
HPST & 1,245 & 2.89 & 10.2\% \\
Kubernetes & 1,298 & 3.01 & 6.4\% \\
Standard FF & 1,387 & 3.22 & Baseline \\
\bottomrule
\end{tabular}
\end{table}

\subsection{Performance Analysis}
\begin{itemize}[leftmargin=1cm]
    \item Response time comparison across strategies
    \item SLA compliance rates
    \item Throughput analysis
    \item Trade-off evaluation
\end{itemize}

\subsection{Device Utilization Patterns}
\begin{itemize}[leftmargin=1cm]
    \item Strategy-specific device preferences
    \item Resource utilization efficiency
    \item Energy consumption by device type
    \item Scaling behavior analysis
\end{itemize}

\section{Discussion}

\subsection{Key Findings}
\begin{itemize}[leftmargin=1cm]
    \item LPLT achieves best energy efficiency (16.8\% savings)
    \item HPST provides optimal performance with energy benefits
    \item Heterogeneous device modeling enables optimization
    \item Smart city deployment feasibility demonstrated
\end{itemize}

\subsection{Practical Implications}
\begin{itemize}[leftmargin=1cm]
    \item Strategy selection guidelines
    \item Deployment recommendations
    \item Energy-performance trade-off insights
    \item Framework extensibility for future research
\end{itemize}

\subsection{Limitations}
\begin{itemize}[leftmargin=1cm]
    \item Simulation-based evaluation constraints
    \item Limited to traffic monitoring use case
    \item Power model based on specifications
    \item Network modeling simplifications
\end{itemize}

\section{Conclusion}

% ==================== FINAL CONCLUSION ====================

\chapter{Conclusion}
\textit{(4-6 pages)}

\section{Summary of Work}
\begin{itemize}[leftmargin=1cm]
    \item Research problem and objectives revisited
    \item State-of-the-art analysis and framework selection
    \item faas-sim extension with energy modeling
    \item Four scaling strategies implementation and evaluation
\end{itemize}

\section{Key Contributions}
\begin{itemize}[leftmargin=1cm]
    \item Extended faas-sim with comprehensive energy modeling
    \item Implemented four distinct energy-aware scaling strategies
    \item Validated through realistic smart city IoT use case
    \item Demonstrated significant energy savings (up to 16.8\%)
    \item Provided practical deployment guidelines
\end{itemize}

\section{Research Impact}
\begin{itemize}[leftmargin=1cm]
    \item Enhanced simulation capabilities for edge-FaaS research
    \item Energy-performance trade-off quantification
    \item Framework available for future research extensions
    \item Practical insights for edge computing deployments
\end{itemize}

\section{Future Work}
\begin{itemize}[leftmargin=1cm]
    \item Real-world testbed validation (Grid'5000 deployment)
    \item Extended use case scenarios beyond traffic monitoring
    \item Machine learning-based predictive scaling
    \item Integration with production FaaS platforms
    \item Advanced energy modeling with thermal considerations
\end{itemize}

\section{Final Remarks}

% ==================== APPENDICES ====================

\appendix

\chapter{Implementation Details}
\begin{itemize}[leftmargin=1cm]
    \item faas-sim framework modification details
    \item Key algorithm implementations
    \item Configuration files and parameters
\end{itemize}

\chapter{Experimental Data}
\begin{itemize}[leftmargin=1cm]
    \item Complete experimental results
    \item Statistical analysis details
    \item Additional performance charts
\end{itemize}

% ==================== PAGE COUNT SUMMARY ====================

\chapter*{Estimated Page Distribution}
\addcontentsline{toc}{chapter}{Page Distribution}

\begin{table}[h]
\centering
\begin{tabular}{lr}
\hline
\textbf{Chapter} & \textbf{Pages} \\
\hline
Chapter 1: General Introduction & 8-10 \\
Chapter 2: Basic Concepts & 18-22 \\
Chapter 3: State of the Art & 15-18 \\
Chapter 4: Our Contribution & 25-30 \\
Chapter 5: Conclusion & 4-6 \\
Appendices & 8-12 \\
\hline
\textbf{Total} & \textbf{78-98 pages} \\
\hline
\end{tabular}
\end{table}

% ==================== KEY RESEARCH CONTRIBUTIONS ====================

\chapter*{Key Research Flow}
\addcontentsline{toc}{chapter}{Research Flow}

\section*{Research Journey}
\begin{enumerate}[leftmargin=1cm]
    \item \textbf{Problem Identification:} Shortage of FaaS simulation tools for edge environments
    \item \textbf{State-of-the-Art Survey:} Analysis of existing simulators and frameworks
    \item \textbf{Framework Selection:} Chose faas-sim as most suitable base framework
    \item \textbf{Gap Analysis:} Identified missing energy modeling capability
    \item \textbf{Energy Model Integration:} Extended framework with power consumption models
    \item \textbf{Scaling Strategy Implementation:} Developed four energy-aware approaches
    \item \textbf{Use Case Validation:} Smart city IoT traffic monitoring scenario
    \item \textbf{Comparative Evaluation:} Energy-performance trade-off analysis
    \item \textbf{Results and Insights:} Demonstrated significant energy savings potential
\end{enumerate}

\section*{Technical Achievements}
\begin{itemize}[leftmargin=1cm]
    \item Successfully extended faas-sim with realistic energy modeling
    \item Implemented comprehensive heterogeneous device modeling
    \item Developed four distinct scaling strategies with clear optimization objectives
    \item Validated framework through realistic smart city deployment scenario
    \item Achieved measurable energy improvements while maintaining performance
\end{itemize}

\end{document}